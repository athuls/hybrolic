Cyberphysical systems are engineered systems that are built from, 
and depend upon, the seamless integration of computational algorithms and physical components.
These systems have entered into every aspect of our lives, and have become an
indispensible part of modern life.
Indeed, the applications of these systems are widespread.
Examples include engineered pacemakers and robotic arms in healthcare, autopilots in
passenger aircrafts and spacecrafts, controllers for chemical plants,
self driving cars, railways systems, critical systems like
nuclear reactors. Clearly the list is endless.

  The framework of Hybrid automata---introduced by Alur, Courcourbetis,
  Henzinger, and Ho~\cite{ACH92}---provides a formal modeling and analysis environment to
  analyze the interaction between the discrete and the continuous parts of
  cyber-physical systems. 
  Hybrid automata can be considered as generalizations of finite state automata
  augmented with a finite set of real-valued variables whose dynamics in each
  state is governed by a system of ordinary differential equations. 
  Moreover, the discrete transitions of hybrid automata are guarded by
  constraints over the values of these real-valued variables, and enable
  discontinuous jumps in the evolution of these variables.
  


Given the immense integration of these systems in modern applicances,
it is natural to ask questions such as whether a given system is free from erros.
Concretely, if a given hybrid system reaches some \emph{error} state, it can lead 
to loss of lives and property. For safety critical systems such as nuclear power plants,
it becomes quite important to verify that the system does not crash. 

As one can expect, there is a rich literature on verification
of hybrid systems to tackle this problem. 
However, it turns out that the  general model of Hybrid automata is quite expressive
and not surprisingly, very simple questions like reachability are undecidable. 
Decidable subclasses for hybrid automata do exist, but most of them rely on a finite bisimulation
to check for reachability, and do not scale very well.

As part of this project, we intend to develop systematic heurisitcs 
for answering  questions such as reachability of some state 
in hybrid systems, using concepts from automated test input generation.
The main advantage for using these techniques is scalability, that is, from
the literature, these techniques look promising, and seem to scale for decently large systems.
Another advantage is that such techniques also give information about
concrete input values that lead the system to a target state.

We will use the model of Singular Hybrid Automata (or \emph{Multi rate automata}
as it was originally called), introduced by Alur et. al., 
to demonstrate the applicability of our approach. 
While the model is quite simple to describe, the reachability problem is
easily undecidable for very simple subclasses (namely when the number of continuous variables
exceed 2).

%%[TODO}: Related Work

The rest of the document is organized as follows.
Section~\ref{sec:sha} describes the details for Singular Hybrid Automata
(referred to as SHA from now on).
Section~\ref{sec:concolic} briefly introduces the framework for
concolic testing and concolic walk.
Section~\ref{sec:approach} descibes our approach, and specific details about
the heuristics we use.
Section~\ref{sec:implementation} describes implementation details about our tool \tool.
%% Section~\ref{sec:experiment} describes the experiments performed with \tool .
The report concludes with Section~\ref{sec:conclusion} .

