The framework of Hybrid automata, introduced by Alur, Courcourbetis, Henzinger, and Ho~\cite{ACH92}, provides a formal modeling and analysis environment to analyze the interaction between the discrete and the continuous parts of cyber-physical systems. Hybrid automata can be considered as generalization of finite state automata augmented with a finite set of real-valued variables whose dynamics in each state is governed by a system of ordinary differential equations. Moreover, the discrete transitions of hybrid automata are guarded by constraints over the values of these real-valued variables, and enable discontinuous jumps in the evolution of these variables. 
In its full generality, the reachability question for Hybrid automata is undecidable. In fact the question remains undecidable for very simple subclasses of Hybrid automata, when the rate of evolution of variables are restricted to be constant integer values, and the constraints in the system are simple linear constraints.
