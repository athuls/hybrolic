The key observation that will aid us in designing the
testing framework is that eventhough the reachability problem asks
for concrete values of time durations in any specific run,
it suffices to only consider at the sequences of transitions taken and theremodes visited.

\subsection{Symbolic Run}
A \emph{symbolic finite run} of an SHA $\Hh$ is a finite sequence 
 \[
 r = \seq{m_0, a_1, m_1, a_2, \ldots m_k}
 \] 
 such that $m_0 \in M_0$ and for all $0 \leq i < k$ we have
 that there are valuations $\nu_i$ and $\nu_{i+1}$, and $t_{i+1} \geq 0$ such that
 $((m_i, \nu_i), (t_{i+1}, a_{i+1}), (m_{i+1}, \nu_{i+1}))$ is a transition
 of $\Hh$.

A \emph{symbolic finite run} of an SHA $\Hh$ can be similarly defined.

Now, given a finite sequence
\[
 r = \seq{m_0, a_1, m_1, a_2, \ldots m_k}
 \] 
it is possible to determine if $r$ is a symbolic finite run or not.

This can be checked by solving a set of linear constraints as described in the following subsection.

\subsection{Checking for Valid Symbolic Runs}
 
Observe that for every (finite or infinite) run 
\[
  r = \seq{(m_0, \nu_0), (t_1, a_1), (m_1, \nu_1), \ldots}
\]
of a WSHA we have
that $\varrho(m_i) \leq \varrho(m_j)$ for every $i \leq j$.
We define the type $\Gamma(r)$ of a finite run 
\[
r = \seq{(m_0, \nu_0), (t_1, a_1), (m_1, \nu_1), \ldots, (m_k, \nu_k)}
\]
as a finite sequence of ranks (natural numbers) and actions  
$\seq{n_0, b_1,  n_1,\ldots, b_p, n_p}$ defined inductively in the following manner: 
\begin{eqnarray*}
  \Gamma(r) = 
\begin{cases}
  \seq{\varrho(m_0)} & \text{ if $r = \seq{(m_0, \nu_0)}$}\\
  \Gamma(r') \oplus (a, \varrho(m)) & \text{ if $r = r'::\seq{(t, a), (m, \nu)}$},
\end{cases}
\end{eqnarray*}  
where $::$ is the cons operator that appends two sequences, while 
for a sequence $\sigma = \seq{n_0, b_1, n_1, \ldots, b_p, n_p}$, $a \in \Sigma$, and $n \in
\Nat$ we define $\sigma \oplus (a, n)$ to be equal to $\sigma$ if $n_p = n$ and
$\seq{n_0, b_1, n_1, \ldots, n_p, a, n}$ otherwise.
Intuitively, the type of a finite run gives the (non-duplicate) sequence of
ranks of modes and actions appearing in the run, where action is stored only
when a transition to a mode of higher rank happens. 
We need to remember only these actions since transitions that stay in the modes
of same rank do not reset the variables. 

Since there are only finitely many ranks for a given WSHA, it follows that for
every infinite run 
\[
r = \seq{(m_0, \nu_0), (t_1, a_1), (m_1, \nu_1), \ldots}
\] there
exists an index $i$ such that  for all $j \geq i$ we have that $\varrho(m_i) =
\varrho(m_j)$. 
With this intuition we define the type of an infinite run $r$ as the type of the
finite prefix of $r$ till index $i$.
Let $\Gamma_\Hh$ be the set of run types of a WSHA $\Hh$.

Given initial values of continuous variables and a concrete trace, it is easy to check if it is a valid trace that reaches some state $q$. However coming up with concrete values for time durations in that trace is non-trivial because the possible values of $t_i$'s are uncountable. Therefore it makes sense to look at symbolic traces where $t_i$'s are symbolic variables. 

A very basic idea to solve reachability question using test generation concepts, is to collect constraints on invariants/guards in a given symbolic trace and feed it to a suitable solver to check for feasibility of the constraints. This will give concrete initial values for the input to the system. It is possible to avoid quantifiers from the constraints because the continuous variables evolve in a linear fashion relative to each other. Therefore we expect simple SMT solvers like Yices to be able to handle these constraints. The challenge is in determining correct symbolic traces that lead to a final goal state in hybrid sytems which have loops. This is because one can generate infinite number of symbolic traces as inputs to our algorithm. We will initially focus our attention on bounded step reachability. Another challenging aspect is to be able to deal with non-linear invariants/guards as constraints that most solvers cannot handle. We attempt to attack this problem by guiding symbolic executions using concrete values for symbolic variables. This is primarily where we will use concepts from concolic testing techniques. Further, such non-linear constraints result in quantified constraints which have one or more quanitified variables. This is again difficult for constraint solvers to handle.  
For hybrid systems where continuous variables evolve in a much complex fashion, such as when the rates of evolution are specified in terms of arbitrary differential equations, there are many more challenges that need to be addressed. We will be looking at such problems as part of future work.  
