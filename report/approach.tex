To begin with, we will model embedded systems using Singular Hybrid Automata (SHA), the reachability question for which is known to be undecidable even for 3 vartiables~\cite{HKPV98}. SHA can be described as a set of discrete states and a set of continuous variables. Additionally there is a function $f(s,x)$ which is a constant integer for every discrete state $s$ and continuous variable $x$, that denotes the rate of evolution of $x$ in state $s$. Moreover every state $s$ and discrete transition between any two such states are annotated using a set of linear constraints. The constraints are referred to as \emph{invariants} and \emph{guards} respectively. A sequence of pairs of the form ($s$,$t$) where $s$ is a discrete state and $t$ is positive constant denoting time spent in that state, describes a valid \emph{trace} if all guards and invariants are met throughout the sequence. We say that a state $q$ is reachable, if there is some initial value for the continuous variables and a finite trace $(s_0, t_0), (s_1, t_1),\ldots (s_k,t_k)$ exists such that $s_k = q$. 

\begin{figure}[h]
  \begin{center}
    \scalebox{0.7}{
    \begin{tikzpicture}[->,>=stealth',shorten >=1pt,auto,semithick]
      \tikzstyle{every state}=[fill=gray!20!white,minimum size=3em,rounded rectangle]
      
      \node[initial, initial where=below,state,fill=gray!30] at (0, 0) (m0) {$\begin{array}{c}
          \dot{x_1} = 0 \\ \dot{x_2} = 0 \\ m_0 \end{array}$};
      
      \node[state, label=below:$1< x_2 < 3$] at (3.5, 2) (m1) {$\begin{array}{c}
          \dot{x_1} = 2\\ \dot{x_2} = 2  \\ m_1 \end{array}$} ;
      
      \node[state, label=below:$2 \leq x_1 < 6$] at (3.5, -2) (m2) {$\begin{array}{c} \dot{x_1} = -2 \\ \dot{x_2} = -2 \\ m_2 \end{array}$} ;
      
      \node[state] at (7, 0) (m3) {$\begin{array}{c} \dot{x_1} = -1 \\ \dot{x_2} = -1  \\ m_3 \end{array}$} ;
  
      \path (m0) edge node {$x_1 < 0, a, \set{x_2}$} (m1);
      \path (m0) edge node[] {$x_2 > 0, b$} (m2);
      
      \path (m2) edge node[] {$ x_1 < 22, c$} (m3);
      \path (m1) edge node[] {$d$} (m3);
   
      \path (m3) edge [loop above] node[] {$e$} (m3);
    \end{tikzpicture}
  }
\end{center}
\label{fig:sha}
  \caption{Singular Hybrid Automata with four modes and two variables.} 
\end{figure}

Given initial values of continuous variables and a concrete trace, it is easy to check if it is a valid trace that reaches some state $q$. However coming up with concrete values for time durations in that trace is non-trivial because the possible values of $t_i$'s are uncountable. Therefore it makes sense to look at symbolic traces where $t_i$'s are symbolic variables. 

A very basic idea to solve reachability question using test generation concepts, is to collect constraints on invariants/guards in a given symbolic trace and feed it to a suitable solver to check for feasibility of the constraints. This will give concrete initial values for the input to the system. It is possible to avoid quantifiers from the constraints because the continuous variables evolve in a linear fashion relative to each other. Therefore we expect simple SMT solvers like Yices to be able to handle these constraints. The challenge is in determining correct symbolic traces that lead to a final goal state in hybrid sytems which have loops. This is because one can generate infinite number of symbolic traces as inputs to our algorithm. We will initially focus our attention on bounded step reachability. Another challenging aspect is to be able to deal with non-linear invariants/guards as constraints that most solvers cannot handle. We attempt to attack this problem by guiding symbolic executions using concrete values for symbolic variables. This is primarily where we will use concepts from concolic testing techniques. Further, such non-linear constraints result in quantified constraints which have one or more quanitified variables. This is again difficult for constraint solvers to handle.  
For hybrid systems where continuous variables evolve in a much complex fashion, such as when the rates of evolution are specified in terms of arbitrary differential equations, there are many more challenges that need to be addressed. We will be looking at such problems as part of future work.  
